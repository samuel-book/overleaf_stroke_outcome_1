\begin{multicols}{2}
\section{Introduction}

% Have 5 paragraphs:
% 1. Background - stroke and benefit of IVT & MT
% 2. Problem being addressed: planning services and benefit of IVT and MT under different service models
% 3. What do we know: dichotomised outcome models
% 4. What don't we know: more detailed models on mRS, shift, Utility
% 5. What does this work seek to address and how


% 1. Background - stroke and benefit of IVT & MT

Globally, stroke presents a very significant, and growing, health burden. In 2019, stroke was the second-leading cause of death (11·6\% of total deaths) and the third-leading cause of death and disability combined (5·7\% of total disability-adjusted life years) \cite{feigin_global_2021}. For ischaemic strokes, the cause of the majority of strokes, the disabling effect may be reduced by reperfusion treatment with intravenous thrombolysis (IVT) \cite{emberson_effect_2014} or mechanical thrombectomy (MT) \cite{fransen_time_2016, goyal_endovascular_2016}, with both treatments reducing in effectiveness within hours of stroke onset.

% 2. Problem being addressed: planning services and benefit of IVT and MT under different service models

When considering acute treatment options, ischaemic stroke may be divided into two broad categories: non-large vessel (nLVO) occlusions may be treated with IVT, and large vessel occlusions (LVO) may be treated with IVT and MT. IVT is a medical treatmentment that may be broadly provided in acute stroke units. Provision of MT is a more specialist service requiring expert interventional radiologists providing procedures in a specialist Computed Tomography (CT) lab. For LVO, MT adds significant clinical benefit over IVT alone \cite{fransen_time_2016, goyal_endovascular_2016}. As both IVT and MT are time-dependent, the optimal organisation of acute stroke services requires careful thought; IVT may most rapidly be provided by a patient attending their closest IVT unit. MT may most rapidly be provided by a patient attending their closest MT unit, but this may require them travelling further than their closest IVT unit, thus delaying IVT for the sake of faster MT. Optimal provision of acute stroke services requires consideration of how to balance time to IVT and MT, which will inform ambulances on where it is best to take any individual patient when there is a choice of a closer IVT-only unit, or a more distant MT-capable unit.

% 3. What do we know
As time to both IVT and MT important, but there may be a choice minimising time to IVT or MT, modelling has been used extensively to help predict the outcome of different models of emergency stroke care. For example, Holodinsky and co-workers have modelled the trade-off between time to IVT both as a general model \cite{holodinsky_drip-and-ship_2018} and applied to specific geographies \cite{kamal_geographic_2018}. These models are probabilistic models predicting a dichotomous outcome of either good or bad outcomes of achieving a certain disability threshold, depending on time to treatment. Schlemm and co-workers have extended this type of model to predict changes in disability-adjusted life years, depending on the probability of re-canalisation with IVT and MT given at different times \cite {schlemm_comparative_2019}. Venema and colleagues presented a stroke outcome model that predicts disability-level outcome, depending on time to MT and IVT \cite{venema_personalized_2019}. That particular model was an advancement on previous models, but was limited by the clinical data available and used, such as assuming that the clinical trials on thrombolysis were treating nLVO only, and by a lack of published information of pre-stroke mRS levels of the stroke patient population, which presents the 'best case' improvement in outcome.

% 4. What don't we know: more detailed models on mRS, shift, Utility.  5. What does this work seek to address and how
The work described here seeks to extend, and build on, the above mentioned work in two key ways: 1) We seek to use the latest and most comprehensive data available to produce disability-level outcome models of reperfusion treatment in stroke, and 2) we aim to produce a model with a primary aim of being able to be reproduced by others, using both detailed descriptions of all assumptions and calculations, and parallel publication of code (using Python, a free and open source computer programming language).
\end{multicols}