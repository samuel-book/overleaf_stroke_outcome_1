\begin{multicols}{2}
\section{Methods}

\subsection{Data avilavilibility}

All data and code used for modelling described in this paper is available at \url{https://github.com/samuel-book/stroke_outcome/}.

\subsection{Modified Rankin Scale}

Modified Rankin Scale (mRS) is the most commonly used instrument to describe post-stroke functional outcome \cite{quinn_functional_2009}. Table \ref{tab:mrs} shows a description of the seven levels of mRS along with utilties, from Wang \emph{et al.}\cite{wang_utility-weighted_2020}. Utilities are a measure of the preference or value that an individual or society gives a particular health state, with 1 being full health and 0 being dead (values of less than zero describe states where death is considered preferable).

\end{multicols}


\renewcommand*{\arraystretch}{1.5} % adjust row spacing
\begin{longtable}[]{@{}llr@{}}
\caption{A description of modified Rankin Scale score, with Utilities from Wang \emph{et al.}\cite{wang_utility-weighted_2020}}\\
\toprule
mRS & Description. & Utility\tabularnewline
\midrule
\endhead
0 & No symptoms. & 0.97\tabularnewline
1 & No significant disability: Able to carry out all usual activities,
despite some symptoms. & 0.88\tabularnewline
2 & \makecell[l]{Slight disability: Able to look after own affairs without assistance, but unable to carry \\ out all previous activities.} &
0.74\tabularnewline
3 & Moderate disability: Requires some help, but able to walk
unassisted. & 0.55\tabularnewline
4 & \makecell[l]{Moderately severe disability: Unable to attend to own bodily needs without assistance, \\ and unable to walk unassisted.} & 0.20\tabularnewline
5 & Severe disability: Requires constant nursing care and attention,
bedridden, incontinent. & -0.19\tabularnewline
6 & Dead. & 0.00\tabularnewline
\bottomrule
\label{tab:mrs}
\end{longtable}



\begin{multicols}{2}
\subsection{Outcome estimation}

This model contains mRS outcome distributions for three patient-treatment cohorts: 1) nLVO treated with IVT, 2) LVO treated with IVT, 3) LVO treeted with MT.

For each patient-treatment cohort, we estimate two mRS distributions: one mRS distribution if treatment is given at \emph{t=0} (time of stroke onset), and one mRS distribution if treatment is given at \emph{t=No Effect} (time of no effect of treatment). In order to estimate these two mRS distributions, we use data from reperfusion treatment clinical trials \cite{lees_time_2010, emberson_effect_2014, goyal_endovascular_2016} and stroke admission data from England and Wales (Sentinel Stroke National Audit Programme). To select the relevant patients for each cohort we use the National Institutes of Health Stroke Scale (NIHSS) on arrival as a surrogate to classify patients as nLVO (NIHSS 0-10) or LVO (NIHSS 11+). NIHSS has been shown to have higher accuracy, in separating nLVO and LVO, than other stroke scales (Area under Receiver Operating Characteristic Curve = 0.86 \cite{duvekot_comparison_2021}).

All mRS distributions created are based on the assumption that, following treatment, the mRS distribution will lie between two extremes: 1) reperfusion perfectly restores function, and the resulting mRS distribution is the same as the pre-stroke mRS distribution, and 2) reperfusion treatment fails to restore any function and the resulting mRS distribution is the same as a control untreated population. Both distributions are corrected for excess deaths that may be caused by the treatment (from clinical trials \cite{emberson_effect_2014, goyal_endovascular_2016}). 

The \textit{t=0} mRS distributions are calculated to give the expected mRS distributions if treatment was given immediately after stroke onset, and also include the risk of excess deaths caused by taking the treatment. 

The \textit{t = No Effect} mRS distributions are based on mRS distribution data when patients did not receive any treatment (this represents what will happen if the patient takes the treatment at \textit{t = No Effect}). This data is obtained from the untreated control groups in clinical trials and further adjusted to include the risk of excess deaths caused by taking the treatment.

To predict the effect of treatment at any given time, we assume that the log odds of being in any particular mRS category will vary linearly over time between \textit{t=0} and \textit{t=No Effect} mRS distributions. This form of decay was fitted to clinical trial data for IVT \cite{emberson_effect_2014} and MT \cite{fransen_time_2016}.

Further details are given in the appendix.

\subsection{Geographic modelling}

\subsubsection{Emergency stroke admissions}

Emergency stroke admissions were based on 238,887 emergency stroke admissions recorded in Hospital Episode Statistics in England over 3 years (2013–2015). A multiple regression model was fitted based on data from the Office of National Statistics: 1) Index of multiple deprivation, 2) The number of people in each Middle Layer Super Output Area (MSOA) in three age bands (0-64, 65-79, and 80+), and 3) The number of people in each MSOA in three health bands (good, fair, bad). Predicted admissions per MSOA are then distributed across their corresponding Lower Super Output Areas (LSOAs) weighted by population.

\subsubsection{Modelling outcome based on patient location}

Geographic modelling is based on emergency stroke units that were active in 2022. All emergency stroke units provide IVT. 25 dedicated neuroscience centres provide MT as well as IVT.

For geographic modelling, we assume the ambulance arrives at the patient 60 minutes after stroke onset. The patient is then taken to the closest appropriate unit, with travel times estimated using Open Street Map.

\end{multicols}

